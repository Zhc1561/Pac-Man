\documentclass[14pt, oneside]{altsu-report}

\worktype{Отчёт по практике на тему:}
\title{Разработка игры Pac-man на языке программирования С++}
\author{Д.\,А.~Кулиев}
\groupnumber{5.205-1}
\GradebookNumber{1337}
\supervisor{И.\,А.~Шмаков}
\supervisordegree{ст.пр}
\ministry{Министерство науки и высшего образования}
\country{Российской Федерации}
\fulluniversityname{ФГБОУ ВО Алтайский государственный университет}
\institute{Институт цифровых технологий, электроники и физики}
\department{Кафедра вычислительной техники и электроники}
\departmentchief{В.\,В.~Пашнев}
\departmentchiefdegree{к.ф.-м.н., доцент}
\shortdepartment{ВТиЭ}
\abstractRU{В отчёте содержатся сведения о курсовой работе.

Курсовая работа заключается в создании кроссплатформенной программы, представляющей собой игру Pac-man.

В отчёте приведено описание используемой библиотеки и инструментария для написания программы на языке программирования С++.

Также в отчёте приведены общие сведения об игре Pac-man.

Наряду с этим представлены проверка работспособности программы, диаграмма UML, полный код программы.}
\abstractEN{The report contains information about the course work.

The course work is to create a cross-platform program that is a Pac-man game.

The report describes the library and tools used to write a program in the C++ programming language.

The report also provides general information about the Pac-man game.

Along with this, a program health check, a UML diagram, and the full program code are presented.}
\keysRU{программа, пакман, игра, кроссплатформенная программа}
\keysEN{program,pacman,game,cross-platform program}

\date{\the\year}

% Подключение файлов с библиотекой.
\addbibresource{graduate-students.bib}

% Пакет для отладки отступов.
%\usepackage{showframe}

\begin{document}
\maketitle

\setcounter{page}{2}
\makeabstract
\tableofcontents

\chapter*{Введение}
\phantomsection\addcontentsline{toc}{chapter}{ВВЕДЕНИЕ}

\textbf{Актуальность:}
Разработка кроссплатформенной программы на языке программирования С++ может послужить хорошим стимулом как для более глубокого изучения программирования, так и для понимания сборки программы под различные платформы.

Разработка игр требует применения широкого спектра навыков программирования, таких как разработка алгоритмов, работа с графикой и обработка ввода.
Программисты могут получить практический опыт в решении реальных проблем и применении своих знаний в контексте.

\textbf{Цель:}
Создать кроссплатформенную игру Pac-man с использованием языка программирования C++ и библиотеки ImGui.

\textbf{Задачи:}
\begin{enumerate}
	\item Изучить графическую библиотеку ImGui.
    \item Реализовать ядро игры Pac-man
	\item Изучить алгоритмы игры Pac-man.
	\item Изучить принцип ООП в C++.
    \item Изучить кроссплатформенную библиотеку для обеспечения совместимости с различными операционными системами
	\item Изучить сборку программы на языке C++.
\end{enumerate}

% Подключение первой главы (теория):
\include{chapter-1-theory.tex}
% Подключение второй главы (практическая часть):
\include{chapter-2-practice.tex} 
\include{chapter-3-test.tex}

\chapter*{Заключение}
\phantomsection\addcontentsline{toc}{chapter}{ЗАКЛЮЧЕНИЕ}

\newpage
\phantomsection\addcontentsline{toc}{chapter}{СПИСОК ИСПОЛЬЗОВАННОЙ ЛИТЕРАТУРЫ}
\printbibliography[title={Список использованной литературы}]

\appendix
\newpage
\chapter*{\raggedleft\label{appendix1}Приложение}
\addcontentsline{toc}{chapter}{Приложение}

\begin{code}
	\captionof*{listing}{Главный файл игры Pac-Man на языке \textit{С++} с использованием библиотеки \textit{ImGui}~\pageref{pysdl2}.}\label{code:main}
	%\inputminted[mathescape,linenos,frame=lines,breaklines]{C++}{src/main.ccp}
\end{code}


\end{document}

